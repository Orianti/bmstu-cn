Подсеть №1: 192.168.15.0/24

Подсеть №2: 192.168.16.0/24

Подсеть №3: 192.168.17.0/24

\section*{Задание 1}

Предварительно был открыт порт маршрутизатора в сети, выполнена команда \code{\# show vlan} для просмотра списка существующих виртуальных сетей. Существующие виртуальные сети являются сетями по умолчанию.

На хостах были настроены адреса интерфейсов.

На коммутаторе были выполнены следующие команды для настройки vlan 10:
\begin{lstlisting}[numbers=none]
Switch#conf t
Switch(config)#int vlan 10
Switch(config-if)#int range fa0/1-2
Switch(config-if-range)#switchport mode access
Switch(config-if-range)#switchport access vlan 10
\end{lstlisting}

Аналогично были настроены vlan 20 и 30.

Для настройки интерфейса GigabitEthernet0/1, к которому подключен маршрутизатор были выполнены следующие команды:
\begin{lstlisting}[numbers=none]
Switch(config)#int g0/1
Switch(config-if)#switchport mode trunk
\end{lstlisting}

\section*{Задание 2}

Для настройки маршрутизатора были выполнены следующие команды:
\begin{lstlisting}[numbers=none]
Router(config)#conf t
Router(config)#int g0/0/0
Router(config-if)#ip address 192.168.1.1 255.255.255.0
Router(config-if)#int g0/0/0.1
Router(config-subif)#encapsulation dot1q 10
Router(config-subif)#ip address 192.168.15.254 255.255.255.0
Router(config-if)#int g0/0/0.2
Router(config-subif)#encapsulation dot1q 20
Router(config-subif)#ip address 192.168.16.254 255.255.255.0
Router(config-if)#int g0/0/0.3
Router(config-subif)#encapsulation dot1q 30
Router(config-subif)#ip address 192.168.17.254 255.255.255.0
\end{lstlisting}