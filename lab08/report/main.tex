Подсеть №1: 192.168.15.0/24

Подсеть №2: 192.168.16.0/24

Подсеть №3: 192.168.17.0/24

Подсеть №4: 192.168.18.0/24

Подсеть №5 (в задании 2): 192.168.25.0/24

\section*{Задание 1}

Предварительно в CLI маршрутизаторов Router0, Router1 и Router2 были выполнены команды \code{\# show ip protocols} и \code{\# show ip rip database}  для проверки наличия уже существующих записей. Записей обнаружено не было.

На хостах были настроены адреса интерфейсов и адреса шлюзов по умолчанию. На маршрутизаторах торах были установлены адреса интерфейсов.

На маршрутизаторе Router0 были выполнены следующие команды:
\begin{lstlisting}[numbers=none]
Router# conf t
Router(config)# route rip
Router(config-router)# network 192.168.16.0
Router(config-router)# version 2
\end{lstlisting}

На маршрутизаторе Router1 были выполнены следующие команды:
\begin{lstlisting}[numbers=none]
Router# conf t
Router(config)# route rip
Router(config-router)# network 192.168.17.0
Router(config-router)# version 2
\end{lstlisting}

На маршрутизаторе Router2 были выполнены следующие команды:
\begin{lstlisting}[numbers=none]
Router# conf t
Router(config)# route rip
Router(config-router)# network 192.168.16.0
Router(config-router)# network 192.168.17.0
Router(config-router)# version 2
\end{lstlisting}

\section*{Задание 2}

Предварительно были открыты порты всех маршрутизаторов в сети, в CLI маршрутизаторов были выполнены команды \code{\# show ip protocols} и \code{\# show ip rip database}  для проверки наличия уже существующих записей. Записей обнаружено не было.

На хостах были настроены адреса интерфейсов и адреса шлюзов по умолчанию. На маршрутизаторах были установлены адреса интерфейсов.

На маршрутизаторе Router0 были выполнены следующие команды:
\begin{lstlisting}[numbers=none]
Router# conf t
Router(config)# route ospf 1
Router(config-router)# network 192.168.15.0 0.0.0.255 area 1
Router(config-router)# network 192.168.25.0 0.0.0.255 area 0
\end{lstlisting}

На маршрутизаторе Router1 были выполнены следующие команды:
\begin{lstlisting}[numbers=none]
Router# conf t
Router(config)# route ospf 1
Router(config-router)# network 192.168.16.0 0.0.0.255 area 2
Router(config-router)# network 192.168.25.0 0.0.0.255 area 0
\end{lstlisting}

На маршрутизаторе Router2 были выполнены следующие команды:
\begin{lstlisting}[numbers=none]
Router# conf t
Router(config)# route ospf 1
Router(config-router)# network 192.168.17.0 0.0.0.255 area 3
Router(config-router)# network 192.168.25.0 0.0.0.255 area 0
\end{lstlisting}

На маршрутизаторе Router3 были выполнены следующие команды:
\begin{lstlisting}[numbers=none]
Router# conf t
Router(config)# route ospf 1
Router(config-router)# network 192.168.18.0 0.0.0.255 area 4
Router(config-router)# network 192.168.25.0 0.0.0.255 area 0
\end{lstlisting}

С помощью команды \code{\# show ip ospf neighbo} установлено, что Router3 определен как BDR, а Router2 --- как BDR. Все маршрутизаторы являются пограничными.

Для настройки аутентификации на каждом маршрутизаторе выполнены следующие команды:
\begin{lstlisting}[numbers=none]
Router# conf t
Router(config)# interface GigabitEthernet0/0/0
Router(config-if)#ip ospf authentication-key qwerty
Router(config)# interface GigabitEthernet0/0/1
Router(config-if)#ip ospf authentication-key qwerty
Router(config-if)#ex
Router(config)#route ospf 1
Router(config-router)#area 0 authentication
\end{lstlisting}